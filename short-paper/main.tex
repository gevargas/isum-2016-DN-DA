%%%%%%%%%%%%%%%%%%%%%%% file typeinst.tex %%%%%%%%%%%%%%%%%%%%%%%%%
%
% This is the LaTeX source for the instructions to authors using
% the LaTeX document class 'llncs.cls' for contributions to
% the Lecture Notes in Computer Sciences series.
% http://www.springer.com/lncs       Springer Heidelberg 2006/05/04
%
% It may be used as a template for your own input - copy it
% to a new file with a new name and use it as the basis
% for your article.
% 
% NB: the document class 'llncs' has its own and detailed documentation, see
% ftp://ftp.springer.de/data/pubftp/pub/tex/latex/llncs/latex2e/llncsdoc.pdf
%
%%%%%%%%%%%%%%%%%%%%%%%%%%%%%%%%%%%%%%%%%%%%%%%%%%%%%%%%%%%%%%%%%%%
   
\documentclass{llncs}

\usepackage{amssymb}
\setcounter{tocdepth}{3}
\usepackage{graphicx}
 
% \usepackage{geometry}
% %\geometry{letterpaper}                   % ... or a4paper or a5paper or ...
% 
% \usepackage{xspace}
 
% \usepackage{epstopdf}
% \usepackage{graphicx,color}
% 
% 
\usepackage{booktabs}
\usepackage{datatool}
\usepackage{tikz}
\usepackage{pgfplots}
\usepackage{pgfplotstable}
\usetikzlibrary{patterns}
\usepackage{lscape}
\usepackage{subfig}

\usepackage{url}
\urldef{\mailsa}\path|{alfred.hofmann, ursula.barth, ingrid.haas, frank.holzwarth,|
\urldef{\mailsb}\path|anna.kramer, leonie.kunz, christine.reiss, nicole.sator,|
\urldef{\mailsc}\path|erika.siebert-cole, peter.strasser, lncs}@springer.com|    
\newcommand{\keywords}[1]{\par\addvspace\baselineskip
\noindent\keywordname\enspace\ignorespaces#1}

\begin{document}

\mainmatter  % start of an individual contribution

% first the title is needed
\title{Can Data Integration Quality be Enhanced on Multi-cloud using SLA?}

% a short form should be given in case it is too long for the running head
%\titlerunning{Lecture Notes in Computer Science: Authors' Instructions}

% the name(s) of the author(s) follow(s) next
%
% NB: Chinese authors should write their first names(s) in front of
% their surnames. This ensures that the names appear correctly in
% the running heads and the author index.
% 
\author{
 Daniel A. S. Carvalho\inst{1},
 Pl\'acido A. Souza Neto\inst{3},  
        Genoveva Vargas-Solar\inst{4},
          Nadia Bennani\inst{2},
        Chirine Ghedira\inst{1}       
}
%
\authorrunning{Lecture Notes in Computer Science: Authors' Instructions}
% (feature abused for this document to repeat the title also on left hand pages)

% the affiliations are given next; don't give your e-mail address
% unless you accept that it will be published
\institute{Universit\'e Jean Moulin, Lyon 3 MAGELLAN, IAE -- France \\
			\email{daniel.carvalho@univ-lyon3.fr, chirine.ghedira-guegan@univ-lyon3.fr}
		\and
		CNRS INSA-Lyon, LIRIS, UMR5205 -- France\\
			\email{nadia.bennani@insa-lyon.fr}
		\and
		Instituto Federal do Rio Grande do Norte, Natal -- Brazil \\
			\email{placido.neto@ifrn.edu.br}
		\and
		CNRS, LIG-LAFMIA, Saint Martin d'H\`eres -- France \\
			\email{genoveva.vargas@imag.fr}
		}

%
% NB: a more complex sample for affiliations and the mapping to the
% corresponding authors can be found in the file "llncs.dem"
% (search for the string "\mainmatter" where a contribution starts).
% "llncs.dem" accompanies the document class "llncs.cls".
%

\maketitle

 
 \begin{abstract}
 The abstract should summarize the contents of the paper and should
 contain at least 70 and at most 150 words. It should be written using the
 \emph{abstract} environment.
 
 \keywords{We would like to encourage you to list your keywords within
 the abstract section}
 \end{abstract}



\bibliographystyle{plain}
\bibliography{bibliography,biblio,example}


\end{document}